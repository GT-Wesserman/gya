
\begin{abstract}
   Altermagnets is a new type of magnetic material which combines some properties of ferromagnets and antiferromagnets. In this project, the magnetic response of the CrSb sample in the x, y and z directions as a function of temperature and magnetic field was measured. By measuring the difference in susceptibility along different axes, the degree of anisotropy was determined. The difference in susceptibility depending on whether Zero Field Cooled or Field Cooled measurements were done also revealed more about the material. By fitting the temperature dependence of susceptibility to a modified Curie-Weiss Law, the Curie constant was extracted which later allowed the finding of the magnetic moment of the involved ions in the material. This in turn was compared to the magnitude of the moments of the ions when viewed in isolation. The altermagnetic material assessed was a CrSb crystal. The results show clear magnetic anisotropy with the b-axis being the preferred magnetization axis. ZFC-FC splitting only appeared along the a-axis with no splitting on the b-axis or c-axis. The difference in magnetic moments between the measurement and the one predicted by theory suggests that itinerant magnetism is a significant contributing factor to the susceptibility of CrSb. The M vs. H measurements showed no strong signs of hysteresis. More sophisticated models need to be used in future experiments to further characterize CrSb.
\end{abstract}