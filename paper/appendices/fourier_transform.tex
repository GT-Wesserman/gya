\section{Fourier Transform}
The Fourier Transform (FT) is a method which allows one to decompose a function into a, sometimes infinite, sum[or  should I write Intergral here?] of simple sinisodul functions. Specifically, the FT takes a function as an input and outputs a new function with the independant variable of frequency where puting the frequency into the function gives a value which represents how much of that frequency is present within the original function. This has shown to have many useful application and as such a basic understanding of the FT gives deeper insight into some of these fields.

\subsection{Derivation of the Formula for the Fourier Transform}
Consider a pure sinusoidal function of time, $t$, with a specific frequency. If one were to multiply this function with the function $e^{-2\pi ft}$ it could be graphed as a vector in the complex plane rotating clockwise around the origin with length equal to the value of our sinusodial function at that time. If the sinusoidal function has a period of 4 seconds, and our $e^{-2\pi ft}$ term a period of 2 seconds, then
