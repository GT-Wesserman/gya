\section{Introduction}
    The existence of a new type of magnetism, coined altermagnetism, was first experimentally demonstrated in 2024 \cite{krempasky2024altermagnetic}. Combining two different states of permanent magnetism, altermagnetism shows promise for having use within the evolving field of spintronics, potentially aiding in creating new and more efficient types of devices. However, due to the recency of their discovery, more experiments need to be made in order to understand the subtleties of how altermagnets work, which materials are the best candidates for creating altermagnets, and how these materials are best used \cite{vsmejkal2022emerging}. This can be done by measuring the magnetic susceptibility of the magnet, which quantifies how much the magnet contributes to an externally applied magnetic field \cite{mugiraneza2022tutorial}.   
    \subsection{CGS emu units}
    One important thing to note is that the field of magnetism within solid state physics usually uses the CGS measurement system. Unlike the SI-system, where the units for mass, time, and length are kilograms, seconds and meters, CGS uses the units grams, seconds, and centimeters as the base units and dervives as many other units as possible from them. There are some different types of CGS unit systems, depending on which physical relationship is used to define them. For example, one could define the unit for magnetic field strength through the field strength from an infinite wire with a current of 1 at a distance of 1 away from it, or define it through the force that a charge of 1 feels while traveling at a speed of 1, all in the previously derived or defined units. For this report the CGS emu system, or Gaussian unit system is used, following the conventions of the field. Thus, magnetic moment will be measured in emu and magnetic field strength in Oersted (Oe). The conversion between emu, which is made up of the units erg/Oe, and Wb$\cdot$m is found with the formula $1 \text{emu} = 4\pi \cdot 10^{-10} \text{Wb}\cdot\text{m}$. It shall be noted that the magnetic field strength and magnetic flux density are not equvivelent, but still higly related. If we still whish to convert between Oersted and Tesla, despite Tesla measuring the magnetic flux density, one would find the conversion $1 \text{Oe} = 1\cdot 10^{-4} \text{T}$ to hold in a vaccum. The difference between magnetic field strength and magnetic flux density is the influence of magnetization of materials inside the field. They are related by the formulas $B=\mu_0(H+M)$ or $B=\mu H$, depending on whether one views the field from magnetization as a seperate field to be added, or as a proportional function of the external magnetic field. 
    \subsection{Crystal Structures, Reciprocal Space and Fermi Levels}
    A crystal is a substance made upf of many atoms arranged in a crystal lattice. This means that there is a periodicity within it, and the pattern of the crystal is repeated througout all of it. The elements making up the crystal are tightly bound to eachother and thus exetr considerable influence on eachother. This influence can, for example, be seen in how the elctrons within the material behave. Some of the elecrtons are localized, meaning they stay close to a specific atom. Other electrons, however, are delocalized, meaning they can more easily move througout the crystal structure, without being completly bound a single specific nuclei. All of these particles are governed by what is called the Schrödinger equation. The Schrödinger equation is a differential equation describing the wavefunction for any particle. While more information about the schrödinger equation can be found in appendix \ref{appendix:schrödinger}, the most important points for this report are that there is a relation between the function for the position of a particle and its momentum which can be used by taking a Fourier Transform of the space the particle inhabits or can inhabit. Doing this results in the momentum-space, or k-space for short, which is a reciprocal space which the particle can inhabit. Doing this yields useful tools for analysing the particle and its state. Any position within the k-space represents momentum that the particle can have. \\ \\ One property of the k-space is that one may measure how much energy is required for an electron to exist at a certain point of the k-space. One way to do this is the following: The binding energy for an electron is given by the formula \begin{equation}\label{eq:eb-from-p}
        E_b=\frac{p^2}{2m}
    \end{equation}
    where $E_b$ is the binding energy, $p$ the momentum of the particle and $m$ the mass of the particle \cite{suto2011energy}. This equation may be rewritten through the formula $p=\hbar k$ where $k$ is the wavevector of the particle. This is because quantum mechanics, specifically the de Broglie relation \cite{guemez2016principle}, tells us that the momentum of a particle is proportional to its wavevector, with the constant of proportionality being $\hbar$, the reduced Planck's constant. This means that Equation (\ref{eq:eb-from-p}) may be rewritten as\begin{equation}
        E_b = \frac{\hbar^2k^2}{2m}.
    \end{equation}
    This in turn means that for any point in the k-space, the binding energy for an electron may be found. If one then moves through k-space such that the wavevector changes linearly, the binding energy will change quadratically. In other words, the shape of the energy as a function of position in k-space can be expected to be parabolic.
    
    
    %Crystal structures are defined by their periodicity. They are built up from a basis which is repeated over the entire crystal structure, providing some symmetries \cite{solidstatephy}. This periodicity means that describing one smaller part of the crystal can be extended to almost the entire crystal. By studying the electrons within the crystal, one can see that some of them are localised, meaning they generally exist near their associated atomic nuclei, while other electrons are delocalised, meaning the have a greater ability to move throughout the crystal structure. The crystal structure exists within a 3-dimensional real space described by the $x$, $y$ and $z$ axis. By taking the Fourier Transform of the space, the mathematical reciprocal space of the crystal structure is defined, also known as the momentum-space, or k-space for short \cite{Foadi_2008}. In the k-space, the units are in reciprocal length and each position in the k-space represents a wavevector \textbf{k}. A wavevector has the magnitude of the wavenumber which is in turn equal to $|k|=\frac{2\pi}{\lambda}$, where $\lambda$ is the wavelength. The direction of a wavevector is along the direction in which the wave propagates \cite{solidstatephy}. Using the k-space for describing how electrons behave within a crystal structure is useful as the wavevector is proportional to the momentum of the electron, which is used to describe some of its properties \cite{guemez2016principle}. Therefore, this report will use the k-space for explanations describing how magnetism and altermagnets work. \\ \\ One property of the k-space is that one may measure how much energy is required for an electron to exist at a certain point of the k-space. One way to do this is the following: The binding energy for an electron is given by the formula \begin{equation}\label{eq:eb-from-p}
%        E_b=\frac{p^2}{2m}
%    \end{equation}
%    where $E_b$ is the binding energy, $p$ the momentum of the particle and $m$ the mass of the particle \cite{suto2011energy}. This equation may be rewritten through the formula $p=\hbar k$ where $k$ is the wavevector of the particle. This is because quantum mechanics, specifically the de Broglie relation \cite{guemez2016principle}, tells us that the momentum of a particle is proportional to its wavevector, with the constant of proportionality being $\hbar$, the reduced Planck's constant. This means that Equation (\ref{eq:eb-from-p}) may be rewritten as\begin{equation}
%        E_b = \frac{\hbar^2k^2}{2m}.
%    \end{equation}
%    This in turn means that for any point in the k-space, the binding energy for an electron may be found. If one then moves through k-space such that the wavevector changes linearly, the binding energy will change quadratically. In other words, the shape of the energy as a function of position in k-space can be expected to be parabolic.
%    \\ \\
    One important energy level to keep track of in the k-space is the Fermi Level which, among other things, denotes the maximum energy state for an electron at the temperature 0 K \cite{kahn2016fermi}. To visualize the Fermi Level, a 3D model of the k-space can include a surface which denotes the location in the k-space where the Fermi level resides. These surfaces are also known as Fermi surfaces. Visualisations of the different energy levels within the k-space in the form of band structure graphs and an example of a Fermi surface is found in \autoref{fig:band-struc-fermi}. One can also note the parabolic shape of the band structures when below the Fermi level, before it changes direction above the Fermi level. One interesting aspect of the energy levels within the k-space is that, in some materials, different spin orientations have different energy levels required in the same position within the k-space.
    \begin{figure}
    \centering
    \includegraphics[width=0.8\linewidth]{images/band_struc_fermi.jpg}
    \caption{(Left) Graph of the band structure where the y-axis shows the energy and the x-axis different positions in k-space, here denoted by different Greek letters. (Right) Visualisation of the Fermi surface for a compund. Image courtesy of Prof. Yasmine Sassa.}
    \label{fig:band-struc-fermi}
\end{figure}
    %By describing the entire crystal structure mathematically based on the repeating basis and a description of the basis based on vectors, it becomes possible to perform a Fourier Transform on the system. Performing the Fourier Transform yields a mathematical space, known as the reciprocal space, k-space or momentum space, which can be used to analyse some properties of that crystal structure. When describing magnetic materials, it is possible to divide different points within the k-space as having different energy levels that electrons need to have to inhabit.
    \subsection{Ferromagnetism}
    %Magnetism as a result of spin and spin interactions
    %Different types of magnetism(ferro, antiferro, dia), what is altermagnetism and why is it interesting. \\ \\
    %Altermagnetism combines a ferromagnetic with an antiferromagnetic structure. This gives it properties of both types of materials and may thus be useful within spintronics \\ \\
    There are various types of magnetism. The first, and most well known, is ferromagnetism. In a material featuring ferromagnetism, all spins within a magnetic domain of the material are aligned in parallel within the direct space, as seen in \autoref{fig:magnetism-examples}a. This alignment is said to be spontaneous, meaning an external magnetic field does not need to be applied for this alignment to happen \cite{mugiraneza2022tutorial}. The spontaneous alignment causes a spin-splitting within the materials, with one orientation of electron spin being more energetically favourable. This is seen in k-space where the same location in the k-space has two different energy levels, one for spin up, the other for spin down, also seen in \autoref{fig:magnetism-examples}a \cite{song2025altermagnets}. The direction of spin which is favoured is the same within the entire magnetic domain. There can be different causes for this spin-splitting within different materials. If magnetic domains are aligned in the same direction, a macroscopic magnetic field is generated. This is what causes everyday permanent magnets to function. This magnetic ordering can, however, be disrupted if the system contains enough thermal energy. The specific temperature where a ferromagnet loses its spontaneous magnetic ordering is called the Curie temperature. Ferromagnets also possess many other useful properties not directly related to their generated macroscopic magnetic fields. The generated field can be disrupted if one applies a magnetic field in the opposite direction. If one then sweeps the field from strong in one direction before weakening and reaching zero before increasing in the opposite direction, one can see how the magnetic field contributed from the ferromagnet suddenly disappears and contributes in the other direction. The resulting graph thus exhibits a loop, which is called a hysteresis loop. \cite{mugiraneza2022tutorial}
    \subsection{Antiferromagnetism}
    A different type of magnetism is antiferromagnetism. Unlike ferromagnetism, where spins are aligned to constructively give rise to a larger magnetic field, antiferromagnetism entails the spins within a material being ordered in such a way as to make all spins cancel each other. Antiferromagnets do not have any spin splitting. Therefore, one spin orientation is not energetically more or less costly than the other. Looking at the momentum space, one can see that all positions within it have the same energy required for both spin directions. This can all be seen in \autoref{fig:magnetism-examples}b. The antiferromagnetic order within the material only exists below the Néel temperature due to the thermal excitations of atoms disrupting the magnetic order above the Néel temperature. The Néel temperature is analogous to the Curie temperature for ferromagnets. \cite{mugiraneza2022tutorial}
    \subsection{Other Magnetic Effects}
    Two other types of magnetism are paramagnetism and diamagnetism. Unlike ferro- and antiferromagnetism, para- and diamagnetism are not formed spontaneously. Paramagnetism only occurs when a material is exposed to an external magnetic field which induces a weak field within the material in the same direction as the external field. Diamagnetism is similar, with the main difference being that the induced field is in the opposite direction of the applied external field, and the material is thus repelled by the field. The diamagnetic contribution is also often far smaller than any paramagnetic contribution \cite{mugiraneza2022tutorial}. Paramagnetic and diamagnetic interactions can be found in many materials, even non magnetic ones such as water. It is through the diamagnetic contribution of water that frogs have been made to levitate in extremely strong magnetic fields, as seen in \autoref{fig:levitating-frog} \cite{berry1997flying}.
    \begin{figure}
    \centering
    \includegraphics[width=\linewidth]{images/Frog_diamagnetic_levitation.jpg}
    \caption{Image of frog levitating due to the diamagnetic contribution of water in strong magnetic fields. Photo taken by Andre Geim.}
    \label{fig:levitating-frog}
\end{figure}
    %\subsection{Spintronics}
    %A new field of study is the study of so called spintronics. The word comes from combining the words spin and electronics. Spintronics is focused on using properties of spins and their interaction with charge to perform actions more typically associated with using a charges. This entails, among other things, using the spins inside of materials to store information or using moving spins or spin waves to transfer signals in a way that doesn't require an electric signal. To use and manipulate these spins, a magnetic material is required, for these are the ones which, by definition, have any sort of order to the arrangement of their spins. The two main spontenous magnetic types, however, each have their own advantages and drawbacks. The spin-splitting of ferromagnets means that spin polarization can occur, meaning the spins within the materials can be more easily used for varoius purposes. This comes at the cost of significant stray magnetic fields which can disrupt nearby components, combined with a higher sensitivity to other magnetic fields, making ferromagnetic materials unreliable in spintronics. The antiferromagnetic materials on the other hand do not create stray magnetic fields and are more resistent to external influences which could disrupt them, but the lack of spin-splitting within the k-space makes them harder to manipulate in such a way as to make them useful for spintronics.
    \subsection{Altermagnetism}
    Despite their names and properties implying mutual exclusivity, ferromagnetism and antiferromagnetism can in some sense be combined \cite{vsmejkal2022emerging}. The phenomenon of a combination of ferro- and anti-ferromagnetism is found within the newly discovered altermagnets \cite{song2025altermagnets}. Despite sharing the antiferromagnetic properties of no macroscopic magnetic field, resistance to external disturbance and the cancellation of magnetic moments, altermagnets also display properties commonly associated with ferromagnets, such as time-reversal asymmetry and spin-polarization in the band structure \cite{vsmejkal2022emerging}. The k-space of an altermagnet is therefore characterised by having different spin orientations being the least energetically taxing at different points of the momentum space, as seen in \autoref{fig:magnetism-examples}c. This means that altermagnets show promise as a material to be used in spintronic devices due to possessing useful ferromagnetic properties, without sharing the associated instability and tendency to affect nearby materials with stray fields \cite{tamang2025altermagnetism}. Another promising quality of altermagnets is the large quantity of materials which may form altermagnets as well as their relative ease of access, not being far too exotic materials. One such candidate is chromium antimonide (CrSb) which has already shown itself to possess altermagnetic properties \cite{peng2025scaling}. 
    \begin{figure}
    \centering
    \includegraphics[width=0.9\linewidth]{images/magnetism_examples.jpg}
    \caption{Top images showcase spin direction in real space for ferromagnets, antiferromagnets and altermagnets. The $x$-, $y$-, and $z$-axes represent real space. The images below show the band splitting, or lack thereof, for each type of magnet in the k-space (also known as momentum space) \cite{song2025altermagnets}. Here, the $x$- and $y$-axes show position in k-space, while the $z$-axes show the energy. The red and blue bands show the difference in energy for spin up and spin down.}
    \label{fig:magnetism-examples}
\end{figure}
    \subsection{Chromium Antimonide}
    CrSb has been studied for its altermagnetic properties, which can be seen through its spin-splitting near the Fermi Level \cite{yang2024three}. It has also been found that the Néel temperature of CrSb is quite high at 712 K meaning that the material does not need to be cooled below room temperature in order to retain its antiferromagnetic properties \cite{peng2025scaling}. CrSb crystals have been shown to exhibit a NiAs type crystal structure, as seen in \autoref{fig:crsb-structure} \cite{yang2024three}. Previous measurements of the magnetic susceptibility of CrSb crystals have been made, but the magnetic anisotropy of the material has not been as extensively studied \cite{peng2025scaling}.
    \begin{figure}
    \centering
    \includegraphics[width=0.8\linewidth]{images/crsb_structure.png}
    \caption{Crystal structure of CrSb with the spins of the Cr atoms marked. Image from \cite{yang2024three}.}
    \label{fig:crsb-structure}
\end{figure}
    \subsection{Localized vs. Itinerant Magnetism}
    Magnetism from a material can have its source at the level of localized electrons or through special interactions between delocalized electrons. On the localized level, magnetization appears due to these localized electrons, meaning those more or less fixed around a specific atom or ion, being ordered in a specific direction. Itinerant magnetism, involving electron interactions, use the delocalized electrons in a material's collective electron bands. Two examples of itinerant magnetism are Pauli paramagnetism and Landau diamagnetism, both of which concern themselves with the spin direction of conduction band electrons. \cite{mugiraneza2022tutorial}
    \subsection{Magnetic Anisotropy}
    For certain magnetic materials, the orientation of the sample has an effect on the magnetization. This is due to the anisotropic materials having crystal structures that make some directions of magnetic moments require less energy. The magnetization of such materials is thus dependent on the direction in which the magnetic field is applied. Finding which direction is easier to magnetize is important for understanding how to use the material in applications most efficiently. \cite{mugiraneza2022tutorial}
    \subsection{Magnetic Moments}
    A magnetic moment is used when referencing how much individual particles contribute to a magnetic field. This is usually measured in Bohr magnetons, $\mu_B$, which is defined by \begin{equation}\label{eq:b-magnetons-SI}
        \mu_B = \frac{e\hbar}{2m_e}.
    \end{equation}Here, $e$ is the elementary charge, $\hbar$ is the reduced Planck's constant, and $m_e$ is the electron mass. Note that it is convention within this field to use CGS-units instead of SI-units. When calculating the magnetic moment of particles, the effective magnetic moment is used. The effective magnetic moment is related to the Curie constant through the equation: \begin{equation}\label{eq:c-from-eff}
        C=\frac{N_A\mu_{\textrm{eff}}^2}{3k_B}
    \end{equation} where $C$ is the Curie constant, $N_A$ is Avogadro's number, $\mu_{\textrm{eff}}$ is the effective magnetic moment and $k_B$ is the Boltzmann constant. By rearranging Equation (\ref{eq:c-from-eff}) the form \begin{equation}\label{eq:u-eff}
        \mu_{\textrm{eff}} = \sqrt{\frac{3k_BC}{N_A}}.
    \end{equation} is found which means that the effective moment can be found if the Curie constant is known. \\ \\
    The method above can be used for magnetic crystals, but a different formula is used when considering free ions without the interactions between particles. The formula follows \begin{equation}\label{eq:free-ions-moment}
        \mu_{\textrm{eff}}= \sqrt{n(n+2)}\mu_{B}
    \end{equation}
    where $n$ is the number of unpaired electrons.
    \subsection{Curie-Weiss Behaviour}
    One common method for analysing the behaviour and properties of magnetic materials is studying if and when it follows Curie-Weiss Behaviour. The Curie-Weiss law states that the magnetization of a material in a unchanging external magnetic field is inversely proportional to the temperature. Specifically, the Curie-Weiss law states that \begin{equation}\label{eq:curie-weiss-law}
        \chi = \frac{C}{T-\theta_{CW}}
    \end{equation}
    where $C$ is the Curie constant, specific to the material, $T$ is the temperature, $\theta_{CW}$ is the Curie-Weiss temperature and $\chi$ is the magnetic susceptibility of the material \cite{mugiraneza2022tutorial}. The susceptibility, in turn, governs the magnetization of the sample according to the relation \begin{equation}
        M=\chi H
    \end{equation} where $M$ is the magnetization and $H$ the external magnetic field. The Curie-Weiss law is expected to hold when the material is in a paramagnetic state, meaning that the temperature is high enough to disrupt the magnetic ordering in the material. To find whether the Curie-Weiss law is the main factor for magnetization, it is often useful to plot the inverse of $\chi$ as a function of temperature and then observe whether a linear relation is found. Taking the inverse of $\chi$, one obtains the equation \begin{equation}
        \chi^{-1} = \frac{T}{C} - \frac{\theta_{CW}}{C}
    \end{equation}
    is found. The curve, if it follows Curie-Weiss behaviour, will then be linear with slope of $1/C$ and y-intercept of $-\theta_{CW}/{C}$ if the line is extended to $T = 0\; K$. \\ \\ The Curie-Weiss law can still be used on materials with more than just a paramagnetic contribution by adding some terms as follows: \begin{equation}\label{eq:extended-curie-weiss}
        \chi=\frac{C}{T-\theta_{CW}}+\chi_0+aT.
    \end{equation}
    The $\chi_0$ term represents a temperature independent contribution to the susceptibility, while the $aT$ term represents a susceptibility which is linearly dependent on the temperature. With the added complexity of the equation, simply taking the inverse and finding a linear fit is no longer feasible, so more advanced line fitting tools need to be used to find all four unknown parameters.
    \subsection{Aim of This Study} % Or Research Question
    %What was the goal, magnetization measurements of single crystals of CrSb (2 directions) \\ \\
    This report aims to measure the magnetic susceptibility of CrSb as a function of temperature and magnetic field strength and analyse how well it follows the Curie-Weiss law, or if a modified version of it needs to be used. The study also aims to analyse CrSb's anisotropic properties and investigate if any ZFC-FC splitting occurs within the material.



