
\section{Discussion}
    The results found within this report can be used to attempt to characterize the nature of CrSb's altermagnetic properties. The magnetic anisotropy has been determined to exist, with the b-axis being the easiest direction for magnetization. The increase in susceptibility along the b-axis was on average around $7.8\times10^{-6} \textrm{emu/cm³/Oe}$ higher than the susceptibility of the ZFC measurements along the a-axis. The ZFC-FC splitting was observed only in the a-axis where a FC measurement yielded an on average 3.427e-06 emu/cm³/Oe. The b- and c-axes showed no ZFC-FC splitting greater than the error in the data. The splitting in the a-axis suggest minor freezing effects for the alignment of magnetic moments.
    \\ \\
    The difference between the calculated effective magnetic moment for Cr\textsuperscript{3+} ions and the one calculated from the extracted Curie Constant shows that there is itinerant magnetism present within CrSb. If the localized moments were the sole contributors to magnetism, the calculated effective moment would be the same as the one found through the measurements. The Moment vs. Magnetic Field graph in \autoref{fig:MvsH-b} is also consistent with what would be expected from a itinerant antiferromagnet. This means that any attempt to further characterize the magnetic properties of CrSb would require other more sophisticated models which can correctly take the interactions between the electrons into account. The Moment vs. Magnetic Field graph for the c-axis, seen in \autoref{fig:MvsH-c}, needs to be further studied, as it is nonlinear while simultaneously showing no signs of a hysteresis loop. Performing more measurements with a large sample with more thickness in the c-axis would reduce the noise and surface contributions found within this study.
\subsection{Conclusion}
    From the analysis, one can conclude that CrSb crystals are anisotropic from the difference in susceptibility between the axes. Specifically, the b-axis was shown to be an easy axis, which is useful to know for future applications of CrSb as a altermagnetic material. The temperature dependence did not follow a standard Curie-Weiss law, due to the susceptibility increasing over the entire temperature range, while the standard Curie-Weiss law predicts decreasing susceptibility with increasing temperature. This suggests itinerant magnetism or other magnetic contributions such as Pauli-paramagnetism. Using a modified Curie-Weiss law with a temperature independent term and a linear term gave a better fit when used over the 50-300 K range. CrSb has also been shown to display a ZFC-FC splitting, however only over the a-axis.
\subsection{Sources of Error}
    During measurements of the magnetic susceptibility, there will always be a diamagnetic contribution from the sample holder which, by necessity, will move with the sample through the magnetic field. While the diamagnetic contribution is usually small, the small size of the sample itself means that there is a risk that the diamagnetic contribution can not be ignored. 
    \\ \\
    There is also a risk for a residual magnetic field to be present within the PPMS. During the ZFC measurements, a residual magnetic field could interfere with the measurements and not provide a true picture of how a ZFC CrSb sample would behave. Any residual material from earlier experiments could also contribute to thus unaccounted for magnetic field.
    \\ \\
    The fitting of the Magnetization vs. Temperature data could also contain sources of error. Due to the unnatural fit when considering the entire temperature range, a smaller range was chosen instead. However, a different range could have had the potential to provide even better values of the Curie constant. The one used should be sufficiently accurate to use the general results, but for a more exact value for the Curie constant to be extracted, more sophistication in measurements or data analysis needs to be used.
\subsection{Outlook}
    Future studies of CrSb will need to use more sophisticated models than merely band structure theory. A better understanding of the susceptibility of CrSb crystals along the c-axis would be attained if one lessens the contributions of the surface by increasing the width of the sample. A more exact mapping of the a-, b- and c-axes to the crystal structure would also be beneficial as this study merely analyses whether a magnetic anisotropy exist but not along which axes in the crystal structure that the a-, b- and c-axes represent. 